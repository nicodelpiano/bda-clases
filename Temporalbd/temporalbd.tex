\documentclass[12pt]{beamer}

\usepackage{color}
\usepackage[normalem]{ulem}

% Include figure files.
\usepackage{graphicx}

% Author, Title, etc.

\title[Bases de Datos Temporales] 
{%
  Bases de Datos Temporales%
}

\author[Del Piano]
{
  Nicol\'as Del Piano
}

\date
{
Bases de Datos Avanzadas
}

\mode<presentation>{
\usetheme{boxes}
\useinnertheme{}
\usecolortheme{whale}
}
\setbeamercolor{structure}{fg=red!80!black}
\setbeamertemplate{footline}[frame number]
\setbeamerfont{page number in head/foot}{size=\normalsize}

\setbeamercolor{postit}{bg=yellow!50!white}
\setbeamercolor{bluelick}{bg=blue!50!white}

\begin{document}
\frame{\titlepage}
\begin{frame}[Opciones]
\frametitle{Introducci\'on}
\onslide<2->{El tiempo es un aspecto importante para los fen\'omenos del mundo real.}\\
\ \\
\onslide<3->{Los eventos ocurren en momentos de tiempo espec\'ificos.}\\
\ \\
\onslide<4->{A veces, nos interesa saber con cierta certeza cu\'ando ocurri\'o tal evento, y poder compararlo con otros para obtener informaci\'on de inter\'es.}
\end{frame}

\begin{frame}[Opciones]
\frametitle{Introducci\'on}
\onslide<2->{Muchas de las \'areas donde se aplican las bases de datos tienen naturaleza temporal: }\\
\begin{itemize}
\onslide<3->{\item Control de inventario.}\\
\onslide<4->{\item Registros m\'edicos.}\\
\onslide<5->{\item Sistemas de informaci\'on geogr\'afica.}
\onslide<6->{\item Operaciones bancarias.}
\onslide<7->{\item Data Warehousing.}
\onslide<8->{\item Sistemas de control de reservas (aerol\'ineas, hoteles, etc.).}
\onslide<9->{\item Aplicaciones cient\'ificas.}
\onslide<10->{\item ... y muchos aspectos m\'as.}
\end{itemize}
\end{frame}

\begin{frame}
\frametitle{Relaciones no temporales}
\begin{center}
\onslide<2->{\begin{tabular}{|c|c|c|c|}
\hline
Id & Nombre & Estado & Sueldo \\
\hline
1 & Juan & Activo & 5700 \\
\hline
2 & Manuel & Activo & 2300 \\
\hline
\end{tabular}
}
\end{center}
\onslide<3->{Muy comunes en los modelos relacionales.\\
\ \\}
\onslide<4->{
Cada tupla representa un hecho que es verdadero ahora.
Solo hay un estado representable de la base de datos: el actual (\textit{current snapshot}).\\
\ \\}
\onslide<5->{A medida que el tiempo transcurre, los datos se van actualizando y modificando.\\
\ \\}
\onslide<6->{
Con este modelo, perdemos informaci\'on.
}
\end{frame}

\begin{frame}
\frametitle{Problem\'atica}
\onslide<1->Las bases de datos convencionales representan el estado de la informaci\'on en un instante de tiempo dado.\\
\onslide<2->Aunque la base de datos es actualizada, estos cambios son vistos como modificaciones del estado actual y los datos obsoletos son borrados.\\
\onslide<3->Por lo tanto, s\'olo podemos utilizar la informaci\'on actual de la base de datos.\\
\onslide<4->Esto genera un problema cuando queremos responder preguntas involucradas a intervalos de tiempo: \center \textbf{?`Cu\'ales empleados percibieron un aumento el mes pasado?}
\end{frame}

\begin{frame}
\frametitle{?`Qu\'e es una Base de Datos Temporales?}
\onslide<2->{Un \textbf{DBMS Temporal} es un Sistema de Gesti\'on de Base de Datos que proporciona herramientas para el manejo y control de Bases de Datos Temporales.\\
\ \\}
\onslide<3->{
Una \textbf{Base de Datos Temporales} es una Base de Datos que tiene dimensi\'on del tiempo a trav\'es del almacenamiento de \textbf{datos temporales}.\\
\ \\}
\onslide<4->{
Proporcionan un marco que mantiene la historia de los cambios que se produjeron en la fuente de datos.\\
}
\end{frame}

\begin{frame}
\frametitle{?`Qu\'e es una Base de Datos Temporales?}
Est\'an dise\~nadas para la captura de la informaci\'on que var\'ia en el transcurso del tiempo.\\
\ \\
\begin{center}\includegraphics[width=7cm,height=6cm]{bdtemp_linea.jpg}
\end{center}
%\ \\
%El programador o dise\~nador de la Base de Datos no necesita escribir c\'odigo adicional para la informaci\'on que var\'ia con el tiempo.
\end{frame}

\begin{frame}
\frametitle{Datos Temporales}
\onslide<2->{Un dato temporal es un dato convencional al que se le asocia un per\'iodo de tiempo para expresar valores temporales en la base de datos.\\
\ \\}
\onslide<3->{
Este agregado de informaci\'on temporal se denomina \textit{time-stamping}.\\
\ \\}
\onslide<4->{
Al asociar el tiempo con la informaci\'on, es posible almacenar diferentes estados de una base de datos.}
\end{frame}

\begin{frame}
\frametitle{Dimensi\'on del tiempo}
Las Bases de Datos Temporales almacenan dos dimensiones del tiempo:
\begin{itemize}
\item Tiempo V\'alido
\item Tiempo Transaccional	
\end{itemize}

\end{frame}

\begin{frame}
\frametitle{Dimensi\'on del tiempo: Tiempo V\'alido}
\onslide<2->{Representa cu\'ando un hecho tiene validez, i.e, es verdadero en el mundo real.\\}
\ \\
\onslide<3->{
Es independiente de si dicho evento fue registrado o no en la base de datos.	\\
\ \\}
\onslide<4->{
Los Tiempos V\'alidos pueden encontrarse en el pasado, presente o futuro.\\
\ \\}
\onslide<5->{
Todos los eventos tienen asociado un Tiempo V\'alido, pero no necesariamente son registrados en una base de datos. Permiten llevar una historia de la BD.
}
\end{frame}

\begin{frame}
\frametitle{Dimensi\'on del tiempo: Tiempo Transaccional}
\onslide<2->{Registra el per\'iodo de tiempo donde un hecho fue almacenado en la base de datos.\\
\ \\}
\onslide<3->{
Permiten consultas que muestren el estado de la base de datos en un tiempo dado.\\
\ \\}
\onslide<4->{
Est\'a acotado en ambos extremos; la creaci\'on de la base de datos y el tiempo presente.\\
\ \\}
\onslide<5->{
Permiten hacer rollback de los datos.}
\end{frame}

\begin{frame}
\frametitle{Ejemplo}

\onslide<2->El Se\~nor X nace en Springfield el 12 de Mayo de 1956.\\
\onslide<2->\begin{center}\includegraphics[width=1.8cm,height=2.3cm]{senorx.jpg}\end{center}
\onslide<3-> Su padre lo registra el 13 de Mayo de 1956.\\
\ \\
\onslide<4-> Se muda a Arroyos Cipreses el 3 de Agosto de 1980, pero olvida registrarse; lo hace el 16 de Agosto del mismo a\~no.\\
\ \\
\onslide<5-> Muere el 20 de Abril de 2004.

\end{frame}

\begin{frame}
\frametitle{Ejemplo: no temporal}
\begin{center}
\begin{tabular}{|c|c|}
\hline
Nombre & ViveEn\\
\hline
Se\~nor X & Springfield\\
\hline
\end{tabular}
\\
\ \\
\begin{Huge}{$\Downarrow$}\end{Huge}\begin{small}{Update}\end{small}\\
\begin{tabular}{|c|c|}
\hline
Nombre & ViveEn\\
\hline
Se\~nor X & Arroyos Cipreses\\
\hline
\end{tabular}
\\
\ \\
\begin{Huge}{$\Downarrow$}\end{Huge}\begin{small}{Delete}\end{small}\\
\begin{tabular}{|c|c|}
\hline
Nombre & ViveEn\\
\hline
\sout{Se\~nor X} & \sout{Arroyos Cipreses}\\
\hline
\end{tabular}
\end{center}
\end{frame}

\begin{frame}
\frametitle{Ejemplo: Tiempo V\'alido}
\begin{center}
\begin{tabular}{|c|c|c|c|}
\hline
Nombre & ViveEn & Valid-From & Valid-To\\
\hline
Se\~nor X & Springfield & 12-May-1956 & $\infty$\\
\hline
\end{tabular}
\\
\ \\
\begin{Huge}{$\Downarrow$}\end{Huge}\begin{small}{Update}\end{small}\\
\begin{tabular}{|c|c|c|c|}
\hline
Nombre & ViveEn & Valid-From & Valid-To\\
\hline
Se\~nor X & Springfield & 12-May-1956 & 2-Aug-1980\\
\hline
\end{tabular}
\\
\ \\
\begin{Huge}{$\Downarrow$}\end{Huge}\begin{small}{Insert}\end{small}\\
\begin{tabular}{|c|c|c|c|}
\hline
Nombre & ViveEn & Valid-From & Valid-To\\
\hline
Se\~nor X & Springfield & 12-May-1956 & 2-Aug-1980\\
\hline
Se\~nor X & Arroyos Cipreses & 3-Aug-1980 & $\infty$\\
\hline
\end{tabular}
\\
\ \\
\begin{Huge}{$\Downarrow$}\end{Huge}\begin{small}{Update}\end{small}\\
\begin{tabular}{|c|c|c|c|}
\hline
Nombre & ViveEn & Valid-From & Valid-To\\
\hline
Se\~nor X & Springfield & 12-May-1956 & 2-Aug-1980\\
\hline
Se\~nor X & Arroyos Cipreses & 3-Aug-1980 & 20-Apr-2004\\
\hline
\end{tabular}

\end{center}
\end{frame}

\begin{frame}
\frametitle{Ejemplo: Tiempo Transaccional}
\begin{center}
\begin{tabular}{|c|c|c|c|}
\hline
Nombre & ViveEn & Transaction-From & Transaction-To\\
\hline
Se\~nor X & Springfield & 13-May-1956 & $\infty$\\
\hline
\end{tabular}
\\
\ \\
\begin{Huge}{$\Downarrow$}\end{Huge}\begin{small}{Insert}\end{small}\\
\begin{tabular}{|c|c|c|c|}
\hline
Nombre & ViveEn & Transaction-From & Transaction-To\\
\hline
Se\~nor X & Springfield & 13-May-1956 & 16-Aug-1980\\
\hline
Se\~nor X & Arroyos Cipreses & 16-Aug-1980 & $\infty$\\
\hline
\end{tabular}
\\
\ \\
\begin{Huge}{$\Downarrow$}\end{Huge}\begin{small}{Update}\end{small}\\
\begin{tabular}{|c|c|c|c|}
\hline
Nombre & ViveEn & Transaction-From & Transaction-To\\
\hline
Se\~nor X & Springfield & 13-May-1956 & 16-Aug-1980\\
\hline
Se\~nor X & Arroyos Cipreses & 16-Aug-1980 & 20-Apr-2004\\
\hline
\end{tabular}

\end{center}
\end{frame}

\begin{frame}
\frametitle{Base de Datos Bi-Temporal}
\onslide<2->{Incluyen ambos tiempos, V\'alido y Transaccional.\\}
\ \\
\onslide<3->{
Proveen informaci\'on \textbf{hist\'orica} y permiten hacer \textbf{rollback} de los datos.\\
\ \\
\begin{tabular}{|c|c|c|c|c|c|}
\hline
\tiny Nombre & \tiny ViveEn & \tiny Valid-From & \tiny Valid-To & \tiny Transaction-From & \tiny Transaction-To\\
\hline
\tiny Se\~nor X & \tiny Springfield & \tiny 12-May-1956 & \tiny $\infty$ & \tiny 13-May-1956 & \tiny 16-Aug-1980\\
\hline
\tiny Se\~nor X & \tiny Springfield & \tiny 12-May-1956 & \tiny 2-Aug-1980 & \tiny 16-Aug-1980 & \tiny $\infty$\\
\hline
\tiny Se\~nor X & \tiny Arroyos Cipreses & \tiny 3-Aug-1980 & \tiny $\infty$ & \tiny 16-Aug-1980 & \tiny 20-Apr-2004\\
\hline
\tiny Se\~nor X & \tiny Arroyos Cipreses & \tiny 3-Aug-1980 & \tiny 20-Apr-2004 & \tiny 20-Apr-2004 & \tiny $\infty$\\
\hline
\end{tabular}
}
\end{frame}

\begin{frame}
\frametitle{Representaci\'on del tiempo v\'alido y transaccional}
\begin{center}\includegraphics[width=8cm,height=4cm]{medium.jpg}
\end{center}
\end{frame}

\begin{frame}
\frametitle{Extensiones Temporales}
\onslide<2->{Hay dos formas de extender el modelo relacional para especificar requisitos temporales.\\
\ \\}
\onslide<3->{
La forma mostrada en los ejemplos antes mencionados se denomina \textbf{marcaje de tuplas}.\\
\ \\}
\onslide<4->{
$(attr_{1},...,attr_{n}) \rightarrow (attr_{1},...,attr_{n},temp\_attr_{1},...,temp\_attr_{m})$\\
\ \\}
\onslide<5->{
La forma restante es denominada \textbf{marcaje de atributos} y usa atributos multivaluados.\\
\ \\}
\onslide<6->{
Al mismo tiempo que permite la correspondencia 1:1 entre entidades y hechos reales, dificulta las actualizaciones y no cumple la 1NF.}
\end{frame}

\begin{frame}
\frametitle{Operadores de Allen}
\onslide<2->{Necesitamos una forma de comparar datos temporales.\\}
\ \\
\onslide<3->{
Allen (1983) propone un conjunto de operadores temporales l\'ogicos para comparar intervalos de tiempo.\\
\ \\}
\onslide<4->{
El operador es una funci\'on de tipo: $I_{t} \times I_{t} \rightarrow \lbrace True, False \rbrace$.}
\end{frame}

\begin{frame}
\frametitle{Algunos operadores temporales (1)}
\begin{center}
\begin{tabular}{cc}
$I_{1}$\ EQUALS\ $I_{2}$ & \includegraphics[width=5cm,height=2cm]{equals.jpg}\\
$I_{1}$\ BEFORE\ $I_{2}$ & \includegraphics[width=5cm,height=2cm]{before.jpg}\\
$I_{1}$\ AFTER\ $I_{2}$ & \includegraphics[width=5cm,height=2cm]{after.jpg}\\
\end{tabular}
\end{center}
\end{frame}

\begin{frame}
\frametitle{Algunos operadores temporales (2)}
\begin{center}
\begin{tabular}{cc}
$I_{1}$\ BEGINS\ $I_{2}$ & \includegraphics[width=5cm,height=2cm]{begin.jpg}\\
$I_{1}$\ INCLUDES\ $I_{2}$ & \includegraphics[width=5cm,height=2cm]{include.jpg}\\
$I_{2}$\ MEETS\ $I_{1}$ & \includegraphics[width=5cm,height=2cm]{meets.jpg}\\
\end{tabular}
\end{center}
\end{frame}

\begin{frame}
\frametitle{Modelo relacional temporal}
\onslide<2->{
El modelo relacional temporal incorpora la sem\'antica temporal en el modelo relacional.\\
\ \\}
\onslide<3->{
Utiliza marcaje de tuplas.\\
\ \\}
\onslide<4->{
Algunas de las caracter\'isticas que debe ofrecer son:}
\begin{itemize}
\onslide<5->{
\item Un tipo de datos para per\'iodos de tiempo, incluyendo la habilidad de representar per\'iodos infinitos (sin fin).}
\onslide<6->{
\item Soporte para tiempo v\'alido, transaccional y tablas bitemporales.}
\onslide<7->{
\item Tiempo transaccional controlado por el sistema.}
\onslide<8->{
\item Consultas temporales.}
\onslide<9->{
\item Predicados y operadores que act\'uen sobre intervalos de tiempo.}
\end{itemize}
\end{frame}

\begin{frame}
\frametitle{Lenguajes de consulta temporales}
\onslide<2->{Una base de datos temporales es un repositorio de informaci\'on temporal.\\
\ \\}
\onslide<3->{
Un lenguaje de consulta temporal es cualquier lenguaje de consulta para bases de datos temporales.\\
\ \\}
\onslide<4->{
Las propiedades de un lenguaje de consulta temporal son:}
\begin{itemize}
\onslide<5->{
\item Sem\'antica declarativa}
\onslide<6->{
\item Implementaci\'on eficiente}
\onslide<7->{
\item Independencia en la representaci\'on}
\onslide<8->{
\item Expresividad de la consulta}
\end{itemize}
\ \\
\onslide<9->{
Algunos ejemplos importantes son: TSQL, TQuel, HRDM, Backlog, etc\'etera. Muchos otros son derivados de estos.}
\end{frame}

\begin{frame}
\frametitle{DBMS Temporal}
%Los DBMS comerciales como Oracle, Sybase, Informix y O2 no son considerados DBMS Temporales ya que no soportan la administraci\'on de datos temporales.\\
%\ \\
\onslide<2->{Los DBMS Temporales deben respaldar un lenguaje de definici\'on de datos temporales, un sistema de restricciones temporales, un lenguaje de manipulaci\'on de datos y un lenguaje de consultas temporales.\\}
\ \\
\onslide<3->{
Algunos ejemplos de DBMS Temporales son: TimeDB, Oracle Workspace Manager, PostgreSQL, IBM DB2, etc\'etera.} 
\end{frame}

\begin{frame}
\begin{center}
\textbf{HISTORIA Y ESTANDARIZACI\'ON}\\
\ \\
Hacia un modelo de datos temporal
\end{center}
\end{frame}

\begin{frame}
\frametitle{Hacia un modelo de datos temporal}
\onslide<2->{
La comunidad de Bases de Datos Temporales fue prol\'ifica en la producci\'on de modelos de datos temporales y lenguajes de consulta.\\}
\ \\
\onslide<3->{
En los \'ultimos 20 a\~nos, docenas de modelos relacionales de datos temporales fueron propuestos.\\
\ \\}
\onslide<4->{
Richard Snodgrass propuso en 1992 que la comunidad de BDT realice extensiones a SQL.\\}
\end{frame}

\begin{frame}
\frametitle{Proceso de estandarizaci\'on de SQL}
\onslide<2->{INCITS(ANSI) DM32.2 responsable del est\'andar SQL en USA.\\
\ \\}
\onslide<3->{
ISO/IEC JTC 1/SC 32 Data Management and Interchange/WG 3 es un comit\'e responsable del est\'andar SQL internacional.\\
\ \\}
\onslide<4->{
Muchas de las capacidades del est\'andar SQL fueron originadas en USA.\\
\ \\}
\onslide<5->{
La aprobaci\'on tiene un ciclo de 3 a 5 a\~nos.\\
\ \\}
\onslide<6->{
7 versiones del Standard SQL:\\
\ \ \ \ 86(SQL-87), 89, 92(SQL2), SQL:1999(SQL3), SQL:2003, SQL:2008, SQL:2011}
\end{frame}

\begin{frame}
\frametitle{SQL Temporal: primer intento (1995-2001)}
\onslide<2->{X3H2(ahora DM32.2) y WG 3 aprobaron el trabajo SQL/Temporal en 1995.\\
\ \\}
\onslide<3->{
USA fue el primero en hacer la propuesta, a\~nadiendo nuevas extensiones a SQL, basadas en el trabajo de Rick Snodgrass.\\}
\ \\
\onslide<4->{
La propuesta de USA estaba basada en TSQL2, una extensi\'on de SQL-92.\\
\ \\}
\onslide<5->{
UK lanz\'o una propuesta similar. Conflictos entre las propuestas de USA y UK, terminaron en que ANSI e ISO decidieran en cancelar SQL/Temporal en 2001.}
\end{frame}

\begin{frame}
\frametitle{SQL Temporal: segundo intento (2008-2011)}
\onslide<2->{En 2008, un segundo intento se hizo presente. Empez\'o con la aceptaci\'on de la propuesta "system-versioned tables" llevada a cabo por INCITS DM32.2 y ISO/IEC JTC1 SC32 WG3.\\
\ \\}
\onslide<3->{
No se resucit\'o SQL/Temporal, sino se a\~nadi\'o la propuesta a SQL/Foundation.\\
\ \\}
\onslide<4->{
Otra caracter\'istica es "application-time period tables".\\}
\ \\
\onslide<5->{
Las caracter\'isticas temporales en SQL:2011 est\'an inspiradas en las anteriores propuestas hechas en SQL/Temporal, pero con una sintaxis un poco diferente.}
\end{frame}

\begin{frame}
\frametitle{SQL-92}
\onslide<2->{
Inclusi\'on de tipos de datos temporales:\\}
\onslide<3->{
\begin{itemize}
\item DATE (10 posiciones) = YEAR, MONTH, DAY (yyyy-mm-dd)
\item TIME (8 posiciones) = HOUR, MINUTE, SECOND (hh:mm:ss)
\item TIMESTAMP (DATE, TIME, fracciones de segundo y desplazamiento de acuerdo al huso horario est\'andar)
\item INTERVAL\\
\ \ \ Per\'iodo de tiempo. Para incrementar/decrementar el valor actual de una fecha, hora o marca de tiempo.\\
\ \ \ Se califica con YEAR/MONTH o DAY/TIME para indicar su naturaleza.
\end{itemize}}
\end{frame}

\begin{frame}
\frametitle{TSQL2 (1)}
\onslide<2->{Es un lenguaje de consulta temporal consensuado por un comit\'e de grupos de investigaci\'on en bases de datos temporales.\\
\ \\}
\onslide<3->{
Apareci\'o como una extensi\'on para SQL-92.\\
\ \\}
\onslide<4->{
La especificaci\'on incluye ideas y conceptos de Tiempo V\'alido, Tiempo Transaccional y Tablas Bitemporales.\\
\ \\}
\onslide<5->{
Tambi\'en conocido como extensiones ANSI X3.135.-1992 e ISO/IEC 9075:1992.}
\end{frame}

\begin{frame}
\frametitle{TSQL2 (2)}
\onslide<1->{Incluye cuatro tipos de marcas de tiempo v\'alido: intervalos, instantes, per\'iodos y elementos.\\}
\ \\
\onslide<2->{Asociadas a estas marcas existen tres categor\'ias de operadores: extractores, constructores y comparadores.\\
\ \\}
\onslide<3->{
Tiene una variable denominada NOW que indica el momento actual $\rightarrow$ tiempo de referencia.\\}
\end{frame}

\begin{frame}
\frametitle{TSQL2: Extractores}
\begin{tabular}{|l|l|}
\hline
Operaci\'on & Operador\\
\hline
\hline
Extractor de eventos & BEGIN(event)\ END(period)\ BEGIN(element)\\
& END(event)\ END(period)\ END(element)\\
\hline
Extractor de per\'iodos & FIRST(period\ FIRST(element)\\
& LAST(period)\ LAST(element)\\
\hline
\end{tabular}
\end{frame}

\begin{frame}
\frametitle{TSQL2: Constructores}
\begin{tabular}{|l|l|}
\hline
Operaci\'on & Operador\\
\hline
\hline
Constructor de eventos & FIRST(event,\ event)\ \\
& LAST(event,\ event)\\
\hline
Constructor de per\'iodos & PERIOD(event,\ event)\\
& INTERSECT(period,\ period)\\
\hline
Constructor de elementos & INTERSECT(element,\ element)\\
& element $+$ element\\
& element $-$ element\\
\hline
\end{tabular}
\end{frame}

\begin{frame}
\frametitle{TSQL2: Comparadores}
\begin{tabular}{|l|l|}
\hline
Operaci\'on & Operador\\
\hline
\hline
Comparador de eventos & event PRECEDES event\\
& event $=$ event\\
& event OVERLAPS event\\
& event MEETS event\\
& event CONTAINS event\\
\hline
Comparador de per\'iodos & period PRECEDES period\\
& period $=$ period\\
& period OVERLAPS period\\
& period MEETS period\\
& period CONTAINS period\\
\hline
Comparador de elementos & element PRECEDES element\\
& element $=$ element\\
& element OVERLAPS element\\
& element CONTAINS element\\
\hline
\end{tabular}
\end{frame}

\begin{frame}
\frametitle{TSQL2: Operadores de Allen}
\begin{small}
\begin{tabular}{| c | c | c |}
\hline
Operador de Allen & Operador TSQL2\\
\hline
\hline
a BEFORE b & a PRECEDES b\\
\hline
a EQUAL b & a $=$ b\\
\hline
a OVERLAPS b & a OVERLAPS b AND \\ & END(a) PRECEDES END(b)\\
\hline
a MEETS b & END(a) = BEGIN(b)\\
\hline
a DURING b & BEGIN(b) PRECEDES BEGIN(a)\\
& AND\\
& END(a) PRECEDES END(b)\\
\hline
a START b & BEGIN(a) = BEGIN(b)\\
& AND\\
& END(a) PRECEDES END(b)\\
\hline
a FINISH b & BEGIN(b) PRECEDES BEGIN(a)\\
& AND\\
& END(a) = END(b)\\
\hline
\end{tabular}
\end{small}
\end{frame}

\begin{frame}
\frametitle{Ejemplos de consulta TSQL2: Tiempo V\'alido (1)}
$>$\ CREATE TABLE empleado(ename VARCHAR(12), eno INTEGER PRIMARY KEY, cumple DATE);\\
$>$\ CREATE TABLE salario(eno INTEGER REFERENCES empleado(eno), sueldo INTEGER);\\
\ \\
$>$\ INSERT INTO empleado\\
\ \ \ \ \ \ VALUES('Homero', 1, DATE '1955-03-21');\\
$>$\ INSERT INTO empleado\\
\ \ \ \ \ \ VALUES('Lenny', 2, '1956-09-18');\\
\ \\
$>$\ INSERT INTO salario VALUES(1, 2000);\\
$>$\ INSERT INTO salario VALUES(2, 4000);\\

\end{frame}

\begin{frame}
\frametitle{Ejemplos de consulta TSQL2: Tiempo V\'alido (2)}
\begin{small}
$>$\ ALTER TABLE salario ADD VALIDTIME PERIOD(DATE);\\
$>$\ ALTER TABLE empleado ADD VALIDTIME PERIOD(DATE);\\
\ \\
$>$\ INSERT INTO empleado\\
\ \ \ \ \ \ VALUES('Carl', 3, DATE '1955-08-10');\\
$>$\ INSERT INTO salario VALUES(3, 3500);\\
$>$\ COMMIT;
\ \\
\begin{center}
\begin{tabular}{|c|c|c||c|}
\hline
ename & eno & cumple & V\'alido\\
\hline
Homero & 1 & 1955-03-21 & [1995-02-01\ -\ 9999-12-31)\\
\hline
Lenny & 2 & 1956-09-18 & [1995-02-01\ -\ 9999-12-31)\\
\hline
Carl & 3 & 1955-08-10 & [1995-02-01\ -\ 9999-12-31)\\
\hline
\end{tabular}
\begin{tabular}{|c|c||c|}
\hline
eno & sueldo & V\'alido\\
\hline
1 & 2000 & [1995-02-01\ -\ 9999-12-31)\\
\hline
2 & 4000 & [1995-02-01\ -\ 9999-12-31)\\
\hline
3 & 3500 & [1995-02-01\ -\ 9999-12-31)\\
\hline
\end{tabular}
\end{center}
\end{small}

\end{frame}

\begin{frame}
\frametitle{Ejemplos de consulta TSQL2: Tiempo V\'alido (3)}
\textbf{Cambiar el nombre de 'Carl' por 'Carlos'.}\\
\ \\
$>$\ VALIDTIME UPDATE empleado\\
\ \ \ \ \ \ SET ename = 'Carlos'\\
\ \ \ \ WHERE ename = 'Carl';\\
$>$\ COMMIT;\\
\ \\
\begin{center}
\begin{tabular}{|c|c|c||c|}
\hline
ename & eno & cumple & V\'alido\\
\hline
Homero & 1 & 1955-03-21 & [1995-02-01\ -\ 9999-12-31)\\
\hline
Lenny & 2 & 1956-09-18 & [1995-02-01\ -\ 9999-12-31)\\
\hline
Carlos & 3 & 1955-08-10 & [1995-02-01\ -\ 9999-12-31)\\
\hline
\end{tabular}
\end{center}
\end{frame}

\begin{frame}
\frametitle{Ejemplos de consulta TSQL2: Tiempo V\'alido (4)}
\textbf{?`Qui\'en percibi\'o un aumento de sueldo?}\\
\ \\
$>$\ UPDATE salario\\
\ \ \ \ \ \ SET sueldo = 1.05 * amount\\
\ \ \ \ WHERE eno = 3;\\
$>$\ COMMIT;\\
\ \\
$>$\ NONSEQUENCED VALIDTIME SELECT ename\\
\ \ \ FROM empleado AS E, salario AS S1, salario AS S2\\
\ \ \ WHERE E.eno = S1.eno AND E.eno = S2.eno\\
\ \ \ \ \ AND S1.sueldo $<$ S2.sueldo AND \\
\ \ \ \ \ \ \ \ VALIDTIME(S1) MEETS VALIDTIME(S2);\\
\begin{center}
\begin{tabular}{|c|}
\hline
ename\\
\hline
Carlos\\
\hline
\end{tabular}
\end{center}
\end{frame}

\begin{frame}
\frametitle{Ejemplos de consulta TSQL2: Tiempo Transaccional (1)}
\begin{center}
\begin{tabular}{|c|c|c||c|}
\hline
ename & eno & cumple & V\'alido\\
\hline
Homero & 1 & 1955-03-21 & [1995-02-01\ -\ 9999-12-31)\\
\hline
Lenny & 2 & 1956-09-18 & [1995-02-01\ -\ 9999-12-31)\\
\hline
Carlos & 3 & 1955-08-10 & [1995-02-01\ -\ 9999-12-31)\\
\hline
\end{tabular}
\end{center}
$>$\ ALTER TABLE empleado ADD TRANSACTIONTIME;\\
$>$\ COMMIT;
\begin{center}
\begin{tiny}
\begin{tabular}{|c|c|c||c|c|}
\hline
ename & eno & cumple & V\'alido & Transacci\'on\\
\hline
Homero & 1 & 1955-03-21 & [1995-02-01\ -\ 9999-12-31) & [1995-07-01\ -\ 9999-12-31)\\
\hline
Lenny & 2 & 1956-09-18 & [1995-02-01\ -\ 9999-12-31) & [1995-07-01\ -\ 9999-12-31)\\
\hline
Carlos & 3 & 1955-08-10 & [1995-02-01\ -\ 9999-12-31) & [1995-07-01\ -\ 9999-12-31)\\
\hline
\end{tabular}
\end{tiny}
\end{center}
\end{frame}

\begin{frame}
\frametitle{Ejemplos de consulta TSQL2: Tiempo Transaccional (2)}
$>$\ UPDATE empleado\\
\ \ \ \ SET ename = `Homero J.`\\
\ \ \ \ WHERE ename = `Homero`;\\
$>$\ COMMIT;
\begin{center}
\begin{tiny}
\begin{tabular}{|c|c|c||c|c|}
\hline
ename & eno & cumple & V\'alido & Transacci\'on\\
\hline
Homero & 1 & 1955-03-21 & [1995-02-01\ -\ 9999-12-31) & [1995-07-01\ -\ 2014-04-23)\\
\hline
Homero J. & 1 & 1955-03-21 & [1995-02-01\ -\ 9999-12-31) & [2014-04-23\ -\ 9999-12-31)\\
\hline
Lenny & 2 & 1956-09-18 & [1995-02-01\ -\ 9999-12-31) & [1995-07-01\ -\ 9999-12-31)\\
\hline
Carlos & 3 & 1955-08-10 & [1995-02-01\ -\ 9999-12-31) & [1995-07-01\ -\ 9999-12-31)\\
\hline
\end{tabular}
\end{tiny}
\end{center}

\end{frame}

\begin{frame}
\frametitle{Ejemplos de consulta TSQL2: Tiempo Transaccional (3)}
\textbf{?`Cu\'ando trabaj\'o alg\'un empleado por m\'as de seis meses?}\\
\begin{center}
\begin{tiny}
\begin{tabular}{|c|c|c||c|c|}
\hline
ename & eno & cumple & V\'alido & Transacci\'on\\
\hline
Homero & 1 & 1955-03-21 & [1995-02-01\ -\ 9999-12-31) & [1995-07-01\ -\ 2014-04-23)\\
\hline
Homero J. & 1 & 1955-03-21 & [1995-02-01\ -\ 9999-12-31) & [2014-04-23\ -\ 9999-12-31)\\
\hline
Lenny & 2 & 1956-09-18 & [1995-02-01\ -\ 9999-12-31) & [1995-07-01\ -\ 9999-12-31)\\
\hline
Carlos & 3 & 1955-08-10 & [1995-02-01\ -\ 1995-03-31) & [1995-07-01\ -\ 9999-12-31)\\
\hline
\end{tabular}
\end{tiny}
\end{center}
\ \\
\begin{small}$>$\ VALIDTIME AND TRANSACTIONTIME SELECT ename, eno\\
\ \ \ \ FROM empleado\\
\ \ \ \ WHERE INTERVAL(VALIDTIME(empleado) MONTH) $>$\\
\ \ \ \ \ \ \ \ INTERVAL `6` MONTH;\\
\ \\
\end{small}
\begin{center}
\begin{tiny}
\begin{tabular}{|c|c|c||c|c|}
\hline
ename & eno & cumple & V\'alido & Transacci\'on\\
\hline
Homero & 1 & 1955-03-21 & [1995-02-01\ -\ 9999-12-31) & [1995-07-01\ -\ 2014-04-23)\\
\hline
Homero J. & 1 & 1955-03-21 & [1995-02-01\ -\ 9999-12-31) & [2014-04-23\ -\ 9999-12-31)\\
\hline
Lenny & 2 & 1956-09-18 & [1995-02-01\ -\ 9999-12-31) & [1995-07-01\ -\ 9999-12-31)\\
\hline
\end{tabular}
\end{tiny}
\end{center}

\end{frame}

\begin{frame}
\begin{center}
\textbf{ESTADO DEL ARTE}\\
\ \\
Las BDT en la actualidad
\end{center}
\end{frame}

\begin{frame}
\frametitle{Tipos de soluciones disponibles}
\onslide<2->{
Actualmente se pueden manipular los datos temporales de las siguientes formas:}
\begin{itemize}
\onslide<3->{
\item Usar un tipo de datos temporal integrado al DBMS y brindar soporte temporal con aplicaciones.}
\onslide<4->{
\item Implementar un tipo de dato abstracto para el tiempo.}
\onslide<5->{
\item Extender el modelo de datos no temporal a uno temporal.}
\onslide<6->{
\item Generalizar un modelo de datos no temporal en uno temporal.}
\end{itemize}
\end{frame}

\begin{frame}
\frametitle{SQL:2011 (1)}
\onslide<2->{El tiempo transaccional es manejado con \textit{system-versioned tables}, que contienen el per\'iodo de tiempo del sistema, y el tiempo v\'alido es manejado con tablas que contienen \textit{application-time period}.\\}
\ \\
\onslide<3->{
transaction time $\rightarrow$ system time\\
valid time $\rightarrow$ application time\\
\ \\}
\onslide<4->{
Las t\'ecnicas e ideas de TSQL se tuvieron en cuenta por el comit\'e, pero las extensiones sint\'acticas que se hicieron difirieron considerablemente de aquellas propuestas en TSQL.
}
\end{frame}

\begin{frame}
\frametitle{SQL:2011 Application-Time tables}
\onslide<2->{El tipo de datos PERIOD para intervalos de tiempo, sigue no disponible. Es simulado usando pares de instantes (con la sem\'antica [cerrado,abierto)).\\
\ \\}
\onslide<3->{
Application-time period tables son tablas que contienen una cl\'ausula PERIOD, con un nombre del per\'iodo definido por el usuario.\\
\ \\}
\onslide<4->{
Estas tablas contienen tambi\'en dos columnas adicionales para almacenar el tiempo de inicio y fin de un per\'iodo de un dato temporal.}
\end{frame}

\begin{frame}
\frametitle{SQL:2011 System-Versioned tables}
\onslide<2->{Son tablas que contienen una cl\'ausula PERIOD con un nombre de per\'iodo predefinido (SYSTEM\_TIME).\\}
\ \\
\onslide<3->{
Contienen dos columnas adicionales que se refieren a el inicio y fin de una transacci\'on.\\
\ \\}
\onslide<4->{
Ambos valores son seteados por el sistema.\\
\ \\}
\onslide<5->{
Preservan las versiones antiguas de las filas.}
\end{frame}

\begin{frame}
\frametitle{SQL:2011 (2)}
\begin{center}
\begin{tabular}{|c|c|c|c|}
\hline
emp\_name & dept\_id & start\_date & end\_date\\
\hline
John & M24 & 1998-01-31 & 9999-12-31\\
\hline
John & J13 & 1995-11-15 & 1998-01-31\\
\hline
Tracy & K25 & 1996-01-01 & 2000-03-31\\
\hline
\end{tabular}\\
\end{center}
\ \\

\textbf{?`En cu\'al departamento estuvo John el 1 de Diciembre de 1997?}\\
\ \\
SELECT dept\_id\\
FROM empleados\\
WHERE emp\_name = 'John' AND start\_date $\leq$ DATE '1997-12-01' AND end\_date $>$ DATE '1997-12-01';


\end{frame}

\begin{frame}
\frametitle{SQL:2011 (3)}
\begin{center}
\begin{tabular}{|c|c|c|c|}
\hline
emp\_name & dept\_id & start\_date & end\_date\\
\hline
John & M24 & 1998-01-31 & 9999-12-31\\
\hline
John & J13 & 1995-11-15 & 1998-01-31\\
\hline
Tracy & K25 & 1996-01-01 & 2000-03-31\\
\hline
\end{tabular}\\
\end{center}
\ \\

\textbf{?`En que departamento est\'a John ahora?}\\
\ \\
SELECT dept\_id\\
FROM empleados\\
WHERE emp\_name = 'John' AND start\_date $\leq$ CURRENT\_DATE AND end\_date $>$ CURRENT\_DATE;

\end{frame}


\begin{frame}
\frametitle{SQL:2011 (4)}
\textbf{Se borra la 3 fila el 16/4/2014:}\\
\ \\
DELETE FROM empleados\\
WHERE emp\_name = 'Tracy';

\begin{center}
\begin{tabular}{|c|c|c|c|}
\hline
emp\_name & dept\_id & system\_start & system\_end\\
\hline
John & M24 & 1998-01-31 & 9999-12-31\\
\hline
John & J13 & 1995-11-15 & 1998-01-31\\
\hline
Tracy & K25 & 1995-11-15 & 2014-04-16\\
\hline
\end{tabular}\\
\end{center}
\end{frame}


\begin{frame}
\frametitle{SQL:2011 vs TSQL}
\begin{tabular}{|c|c|}
\hline
TSQL & SQL:2011\\
\hline
\textbf{valid} time & \textbf{application} time\\
\textbf{transaction} time & \textbf{system} time\\
\hline
timestamping & versioning\\
\hline
\textbf{validtime} table & \textbf{application time period} table\\
\textbf{transactiontime} table & \textbf{system-versioned} table\\
\textbf{bitemporal table} & \textbf{system-versioned}\\
& \textbf{application time period} table\\
\hline

\end{tabular}
\end{frame}


\begin{frame}
\frametitle{TimeDB}
\onslide<2->{
TimeDB es un DBMS Bitemporal basado en SQL.\\
\ \\}
\onslide<3->{
Soporta un lenguaje de consulta, un lenguaje de manipulaci\'on de datos y un lenguaje de definici\'on de datos.\\}
\ \\
\onslide<4->{
TimeDB brinda ATSQL2, un lenguaje de consulta basado en TSQL2, ChronoLog y Bitemporal ChronoLog.\\
\ \\}
\onslide<5->{
Manipula los datos temporales buscando extender el modelo de datos relacional a uno temporal.\\
\ \\}
\onslide<6->{
Traduce sentencias TSQL en sentencias SQL est\'andar para luego ser ejecutadas en DBMS como Oracle, Sybase, etc\'etera.}
\end{frame}

\begin{frame}
\frametitle{TimeDB: TDDL}
\onslide<2->{En TimeDB, una tabla bitemporal puede crearse de esta manera:\\}
\ \\
\onslide<3->{
CREATE TABLE empleados (EmpID INTEGER, Name CHAR(30), Department CHAR(40),\\
Salary INTEGER)\\
AS VALIDTIME AND TRANSACTIONTIME;\\}
\end{frame}


\begin{frame}
\frametitle{TimeDB: TDML}
\onslide<2->{
Las siguientes sentencias insertan datos temporales a una tabla:\\}
\ \\
\onslide<3->{
VALIDTIME PERIOD '1985-1990'\\
INSERT INTO empleados VALUES (10,'John','Research',11000);\\
\ \\
VALIDTIME PERIOD '1990-1993'\\
INSERT INTO empleados VALUES (10,'John','Sales',11000);\\
\ \\
VALIDTIME PERIOD '1993-forever'\\
INSERT INTO empleados VALUES (10,'John','Sales',12000);\\
}
\end{frame}

\begin{frame}
\frametitle{TimeDB: TQL}
\onslide<2->{Para hacer consultas:\\}
\ \\
\onslide<3->{
VALIDTIME\\
SELECT * FROM empleados;\\
\ \\
TRANSACTIONTIME\\
SELECT * FROM empleados;\\
\ \\
VALIDTIME AND TRANSACTIONTIME\\
SELECT * FROM empleados;}
\end{frame}

\begin{frame}
\frametitle{Oracle Workspace Manager (1)}
\onslide<2->{Es una herramienta de Oracle Database que brinda a los desarrolladores la capacidad de manejar distintas versiones temporales de la base de datos.\\
\ \\}
\onslide<3->{
Se puede habilitar el soporte de tiempo v\'alido al momento de creaci\'on de una tabla:\\}
\begin{tiny}
\begin{itemize}
\onslide<4->{\item Crear la tabla de empleados y sus salarios:\\
\ \ CREATE TABLE empleados (\\
\ \ \ \ \ name VARCHAR(16) PRIMARY KEY,\\
\ \ \ \ \ salary NUMBER\\
\ \ );\\}
\onslide<5->{
\item Versionar la tabla. Especificar TRUE para el soporte de tiempo v\'alido:\\
EXECUTE DBMS\_WM.EnableVersioning('empleados','VIEW\_WO\_OVERWRITE',FALSE,TRUE);
}
\onslide<6->{
\item Insertar las filas:\\
\ \ INSERT INTO empleados VALUES(\\
\ \ \ \ 'Adams',\\
\ \ \ \ 30000,\\
\ \ \ \ WMSYS.WM\_PERIOD(TO\_DATE('01-01-1990','MM-DD-YYYY'),\\
\ \ \ \ \ \ \ \ \ \ \ \ TO\_DATE('01-01-2005','MM-DD-YYYY'))\\}
\end{itemize}
\end{tiny}
\end{frame}

\begin{frame}
\frametitle{Oracle Workspace Manager (2)}
\onslide<2->{El tipo de datos WM\_PERIOD es usado para especificar el rango del tiempo v\'alido.\\
\ \\}
\onslide<3->{
Las constantes del tiempo v\'alido son:\\
DBMS\_WM.MIN\_TIME $\approx$ 01-Jan-(-4712)\\
DBMS\_WM.MAX\_TIME $\approx$ 31-Dec-9999\\
DBMS\_WM.UNTIL\_CHANGED es un TIMESTAMP que se comporta como MAX\_TIME hasta que es modificado.\\
\ \\}
\onslide<4->{
Algunos operadores:\\
WM\_OVERLAPS,\ WM\_CONTAINS,\ WM\_MEETS,\ WM\_EQUALS,\ WM\_INTERSECTION, etc\'etera.}
\end{frame}

\begin{frame}
\frametitle{Bibliograf\'ia}
\begin{itemize}
\item T\'opicos Avanzados de Bases de Datos. Cristina Bender, Claudia Deco.
\item Temporal Databases. Richard T. Snodgrass.
\item Temporal Query Languages: a Survey. Jan Chomicki.
\item Temporal features in SQL:2011. Krishna Kulknarni.
\item Extensions to SQL for Historical Databases. L. Sarda.
\item Introduction to Temporal Database Research. Christian S. Jensen.
\end{itemize}
\end{frame}
\end{document}